\documentclass[final]{beamer}
\usepackage[size=a0,scale=1.4]{beamerposter}
\usetheme{Berlin}
\usecolortheme{rose}

\usepackage{graphicx, amssymb, epstopdf, setspace, url, natbib, semtrans, comment, pdfpages, enumerate, float, tocloft, caption, fancyhdr, rotating, lscape, pdflscape, lipsum, pbox}
\usepackage[english]{babel}
\newcommand\fnote[1]{\captionsetup{font=footnotesize}\caption*{#1}}
\DeclareGraphicsRule{.tif}{png}{.png}{`convert #1 `dirname #1`/`basename #1 .tif`.png}
\usepackage{verbatim, multirow, threeparttable, colortbl, chngcntr, ragged2e, booktabs, tabularx, amsmath}
\newcommand{\btVFill}{\vskip0pt plus 1filll}
\newcommand{\sym}[1]{\rlap{#1}}
\DeclareMathOperator*{\Max}{Max}
\DeclareMathOperator*{\E}{\mathbb{E}}
\usepackage{tabularray}
\usepackage{longtable}
\usepackage{threeparttablex}
\usepackage{threeparttable}
\usepackage[font=small,labelfont=bf]{caption} 


%\renewcommand{\thesection }{\Roman{section}.} 
%\renewcommand{\thesubsection}{\thesection\Alph{subsection}.}
\usepackage{tikz}

\usepackage{tabularray}
\usepackage[titletoc, title]{appendix}
\usepackage{etoolbox}
\patchcmd{\appendices}{\quad}{: }{}{}

\usepackage{textcase}
\usepackage[tablename=Table]{caption}
%\DeclareCaptionTextFormat{up}{\MakeTextUppercase{#1}}
%\captionsetup[table]{
    %labelsep=newline,
    %justification=centering,
    %textformat=up,}
\usepackage[figurename=Figure]{caption}
%\DeclareCaptionTextFormat{up}{\MakeTextUppercase{#1}}
%\captionsetup[figure]{
    %labelsep=newline,
    %justification=centering,
    %textformat=up,}
\usepackage{bbm}   

\usepackage{subcaption}

\renewcommand{\thetable}{\arabic{table}}
\setcounter{table}{0}
\renewcommand{\thefigure}{\arabic{figure}}
\setcounter{figure}{0}

% Custom style for highlighted blocks
\setbeamercolor{highlighted block}{fg=black, bg=yellow} 

\title{\Huge The Long-Term Effects of Africa's Slave Trades (Nathan Nunn, 2008)} % Very large title
\author{\Large IV Group: Madison, Charlie, Davis, Anel} 
\institute{\Large Texas A\&M University} % Larger institute name
\date{\Large\today} % Larger date

% Clearing the default footline
\setbeamertemplate{footline}{} 

\begin{document}
\begin{frame}[t]

% Title at the top
\begin{block}{}
\centering
\maketitle
\end{block}

\begin{columns}[T] % align columns at the top

% Column 1
\begin{column}{.32\textwidth}
    \begin{block}{\Huge Abstract} % Very large section title


    \Large % Larger main text
    The study examines how the slave trades continue to affect Africa's economy. By analyzing historical shipping records and slave data, it reveals that countries exporting more slaves now experience weaker economies. It suggests that the slave trades disrupted economic growth by fracturing communities and political structures.
    \end{block}

    \vspace{1cm} % Add vertical space

    \begin{block}{\Huge Introduction} % Very large section title
    \Large % Larger main text
     Africa's economic struggles in the late 20th century stem from exploitation via slavery and colonialism. Previous studies connect colonialism to underdevelopment but lack empirical research on Africa's slave trades. This paper aims to bridge this gap by estimating exported slaves from Africa and studying their impact on current economic development.
    \end{block}

    \vspace{1cm} % Add vertical space

    \begin{block}{\Huge Literature Review} % Very large section title
    \Large % Larger main text
    This paper discusses how Africa's underdevelopment is influenced by slavery and colonialism, citing Bairoch (1993) and Manning (1990). It connects previous colonial research with a new empirical analysis of the slave trades' direct economic impacts, revealing a gap in understanding their lasting consequences.
    \end{block}
\end{column}

% Column 2
\begin{column}{.32\textwidth}
    \begin{block}{\Huge Methodology} % Highlighted block
    \Large % Larger main text
    It utilizes shipping records and historical documents to estimate the number of slaves exported from each African country during the slave trades. Combining this with contemporary economic data, it examines the correlation between historical slave exports and current economic performance. To assess causality, it employs IV based on geographic distances from major slave markets, ensuring a robust analytical framework for exploring the slave trades' long-term economic effects.
    \end{block}

    \vspace{1cm} % Add vertical space

    \begin{block}{\Huge Findings} % Highlighted block
    \Large % Larger main text
    The findings reveal a significant negative relationship between the number of slaves exported from African countries during the slave trades and their current economic development. Those demonstrate that regions most affected by the slave trades are among the poorest today. 

    \begin{table}[H]
\centering
\caption{Regression Results Summary}
\begin{tabular}{llll}
\toprule
Independent Variable & Coefficient & SE  \\
\midrule
ln(exports/area)& -0.112*** & 0.024 \\
SS ln(exports/area) & -0.208*** & 0.053 \\
FS Atlantic Distance & -1.31*** & 0.357 \\
FS Indian Distance & -1.10*** & 0.380 \\
FS Saharan Distance & -2.43*** & 0.823 \\
FS Red Sea Distance & -0.002 & 0.710 \\
\hline
\bottomrule
\end{tabular}
\end{table}
    
    \end{block}
\end{column}

% Column 3
\begin{column}{.32\textwidth}
    \begin{block}{\Huge Discussion} % Very large section title
    \Large % Larger main text
    The paper stresses the lasting economic impact of Africa's slave trades, suggesting they directly contribute to current underdevelopment. It argues that the negative link between slave exports and economic performance is causal. The study underscores the need to acknowledge historical injustices like the slave trades in understanding present economic disparities in African countries.
    \end{block}

    \vspace{1cm} % Add vertical space

    \begin{block}{\Huge Conclusions} % Very large section title
    \Large % Larger main text
    This study concludes that Africa's slave trades have left a deep, lasting scar on the continent's economic development. It provides empirical evidence linking the historical export of slaves to current economic underperformance. This analysis underscores the need to recognize the enduring impact of the slave trades on Africa's economic landscape, suggesting historical exploitation plays a significant role in contemporary economic disparities.
    \end{block}

    \vspace{1cm} % Add vertical space

    \begin{block}{\Huge References} % Very large section title
    \Large % Larger main text
    Nunn, N. (2008). The Long-Term Effects of Africa's Slave Trades. The Quarterly Journal of Economics Vol. 123, No. 1 (Feb., 2008), 139-176 .
    \end{block}

\end{column}

\end{columns}
    \vspace{1cm} % Add vertical space
% Custom Footer
\begin{beamercolorbox}[center]{section in head/foot}
\end{beamercolorbox}

\end{frame}
\end{document}
